\documentclass{article}

\title{Model Description}
\begin{document}

\maketitle

\section{Model for CMPE 557 Final}
\subsection{Parameters}
\begin{enumerate}
    \item $N$: Number of agents
    \item $M$: Number of meanings
    \item $m$: Memory size
    \item $W$: Number of symbol
\end{enumerate}

\subsection{Interaction}
Currently, a speaker and a listener agent are selected. Speaker agent selects a random meaning $\mu_{s}$. She checks her memory for that meaning: if no symbols are present, she randomly selects one and adds it to her memory. If there is at least one symbol in her memory for $\mu_{s}$, she probabilistically selects a symbol $\sigma$.

She conveys $\sigma$ to the listener agent. Listener understands a meaning $\mu_{l}$ from symbol $\sigma$ with probability $p$
\\
\begin{equation}
    p=\frac{k_{\sigma,\mu_{l}}}{k_{\sigma}}    
\end{equation}


where \\

$k_{\sigma,\mu_{l}}=$ Number of times $\sigma$ is seen in agent's memory for $\mu_{l}$ \\

$k_{\sigma}=$ Number of times $\sigma$ is seen in total in the agent's memory
\\

Then, magically, both parties learn what $\mu_{s}$ and $\mu_{l}$ are.
If $\mu_{s}=\mu_{l}$, the interaction is a success. If not, it is a failure.

\subsection{Memory updates}

If the interaction is a success, both agents update their memory for $\mu_{s}=\mu_{l}$ with $\sigma$

If the interaction is a failure, only the listener agent updates her memory for $\mu_{s}$ with $\sigma$

For a memory with empty slots, updating the memory means putting the symbol on an empty memory slot. For a full memory, we mean writing the symbol in place of the memory slot that has not been updated for the longest time. 

\section{New model considerations}

The old model suffers from a deficiency: it assumes telepathic communication between the two agents. This makes us ask "Why do they even need a language then?"

\subsection{Meanings as actions}

If we interpret meanings as actions such as "Turn right!", "Don't go to east!", then the speaker can understand when the listener has not understood her. This assumes:

\begin{enumerate}
    \item There is no confusion or noise in transmission of symbols
    \item At a time, there is only one listener(L) and one speaker(S)
    \item If L understands S, she will do as S says
\end{enumerate}

The counter-positive statement of the second assumption is "If L doesn't do what S says, then L did not understand". Thus, S can know when L did \textbf{not} understand.

S can't be sure that L understood S because it may be that L did what S told without actually understanding the symbol. L may be moving randomly.

\subsubsection{How to know if L understood me}

One way S can be sure that L understands her is by repeated communication. If S sends the same symbol (or maybe different symbols?) repeatedly and L does the intended action in all(or maybe most?) of them, then S can know L understands her.

Another possibility:
Let's look at the statement: "If L does what S says, she understood S". The counter-positive of this is "If L didn't understand S, she doesn't do what S says" When is this true?

\subsubsection{How to know if I understood S}
Let's take the point of view of L. How do we know if we understood someone? This is the case between a baby and her mother. Baby sends a symbol(crying) and mother does an action. Until the mother does the right action, baby keeps sending the symbol. Thus, again we have a repeated communication. 


\end{document}
